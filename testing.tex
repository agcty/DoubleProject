\chapter[Test Strategy]{Test Strategy} \label{ch:testing}

\section{Introduction}
In order to ensure correct functionality and fulfillment of the requirements a test strategy was implemented. Common ways of testing frontend applications include "Unit Testing", "Snapshot Testing", "End to End Testing" and "Integration Testing". Given the projects size and its time constraints, not all features could be tested, also not every possible testing methodolgy was sensible to be utilized. The project's testing strategy is based on Unit and Snapshot testing, with a special focus on the most import aspects and features of the application. Furthermore, a pre-emptive and heuristic approach was chosen. 

\section{Testing Methodologies}

\subsection{Unit Testing}
The testing strategy consists of a pre-emptive and heuristic approach (see \autoref{preheur}), based on Unit and Snapshot testing.
Unit testing is a way of testing the functional aspects of software components. In Blackbox testing, there is no knowledge of the internals of a component such as data structures, properties or methods. Whitebox testing on the other hand, is closely coupled to a component's internals. The mixture between blackbox and whitebox is called "Greybox Testing", which is also a common methodology \cite{spillner2014software} and used as an approach of Unit for this project.

\subsection{Snapshot Testing}
Snapshot tests are a useful tool for making sure the rendered UI does not change unexpectedly \cite{SnapshotJest:online}.

A very comprehensible explanation is provided by Jest - a Snapshot testing library \cite{SnapshotJest:online}:

\begin{quotation}
"A typical snapshot test case for a mobile app renders a UI component, takes a snapshot, then compares it to a reference snapshot file stored alongside the test. The test will fail if the two snapshots do not match: either the change is unexpected, or the reference snapshot needs to be updated to the new version of the UI component."
\end{quotation}

\subsection{Pre-emptive and Heuristic Approach} \label{preheur}
As there is no way of objectively defining the priority with which a component should be tested from the given requirements, a heuristic approach of testing "which relies on experience and rules of thumb" \cite{spillner2014software} was chosen. Based on my personal experience and recommendations of the Vue community, I concluded that all Single File Components of the /components folder should be tested, whereas pages should not be tested as they are "just" containers for the components. In addition to a heuristic approach, it was sensible to follow a pre-emptive testing approach which means testing is carried out from project start \cite{spillner2014software}. This also complements the iterative development approach and continuous integration. 

\section{Test Setup}
The test set up is built upon two external libraries: "vue-test-utils" - the official testing library for vue applications and "Jest" - a JavaScript testing library. While Jest is used for general JavaScript test scenarios such as "has a button been clicked" or "is a class added when this condition applies", snapshot testing and to generate code coverage, "vue-test-utils" is needed for Vue-specific scenarios and set up. This includes checking if a computed property is has the right value or checking if the underlying Vue instance is created properly.

For every component-group (e.g every component in /Auth folder) a "\_\_test\_\_" folder is created which contains the tests for this group. In general the following tests are executed for every tested component: testing if a component mounts properly on the Vue instance and if it renders properly and matches the stored snapshot. Depending on the component and its functionality, different tests are applied. For example a specific test for the LoginForm.vue component tests the following scneario: if username and password are supplied to the form and both are formatted correctly and the submit button is clicked, the onSubmit() method of the component should have been called. \autoref{fig:testpass} and \autoref{fig:testresults} show the passed tests and test results respectively.

Testing the previous examples may sound trivial but it is important to note that in order to be able to execute a test, the component needs to be embedded into an appropriate environment. Setting up this environment however, differs from component to component. For example the LoginForm.vue component uses the translation plugin, communicates with the Vuex store and requires VeeValidate to validate form input. This means that the entire store (or the needed module of the store), the translation plugin and VeeValidate have to be mocked. The corresponding code sample for these scenarios can be found in \autoref{loginformtest}.

In addition, by default every test uses a global Vue instance which is easily polluted if various tests mock different plugins and libraries. To mitigate this issue, a "local" Vue instance was used which essentially means that every test has its own isolated Vue environment. This makes the test setup significantly more complicated but mitigates the risk of global pollution and results in more comprehensible tests. 

\section{Static Analysis}
The following text was originally published at \href{https://medium.com/@gogl.alex/how-to-properly-set-up-eslint-with-prettier-for-vue-or-nuxt-in-vscode-e42532099a9c}{How to properly set up Nuxt with ESLint and Prettier in VSCode}.
With JavaScript being a dynamic and loosely-typed programming language problems and errors can potentially stay undetected for a very long time. To analyze code and to find errors, JavaScript applications are typically executed. This however, is a complete waste of processing power and hinders a fast development process.

As an alternative, static analysis tools can be used. One very common tool is "ESLint", more specifically, it is a linting tool. Linting is defined as a type of static analysis that can be utilised to detect problematic code patterns or to enforce code style guidelines. This allows programmers to detect potential problems without executing the code. Furthermore, ESLint is a highly flexible and configurable linting tool in which every aspect can be adapted to fit a project's needs.

In addition to unit and snapshot tests, the project incorporates this type of static analysis to ensure that a specific code pattern and guidelines are respected across every part of the project and across developers.

\section{Continuous Integration}
To ensure that the system stays functional, continuous integration is set up. In its basic form, it is a pattern that ensures tests are run everytime the system's code changes. As described in \autoref{ci}, Circle CI is used as an external CI platform. It is set up as a complementary tool for Git and the Git Flow pattern. Anytime a new commit is pushed to the "master" or "develop" branch, Circle CI is notified and then builds and tests the code. As it is based on containers, this setup is very quick. Testing the building stage is crucial as it makes sure there are no syntax errors which result in a broken system. It then runs all unit and snapshot tests in addition to statical analysis of the code with ESLint. \autoref{ciconfig} depicts the configuration of this process. \newline

\begin{lstlisting}[caption=Circle CI configuration, captionpos=b, style=htmlcssjs, label=ciconfig]{Circle CI configuration}
    version: 2
    jobs:
      build:
        docker:
          - image: circleci/node:11.10.1
    
        working_directory: ~/hv-frontend
    
        steps:
          - checkout
          # Download and cache dependencies
          - restore_cache:
              keys:
                - v1-dependencies-{{ checksum "package.json" }}
                # fallback to using the latest cache if no exact match is found
                - v1-dependencies-
    
          - run: npm install
    
          - save_cache:
              paths:
                - node_modules
              key: v1-dependencies-{{ checksum "package.json" }}
    
          # run build and tests!
          - run: npm run build
          - run: npm test
    
    workflows:
      version: 2
      build_and_test:
        jobs:
          - build
    
    
  \end{lstlisting}